
\chapter{Piezas Impresas} % Main appendix title

\label{app:piezas} % For referencing this appendix elsewhere, use \ref{AppendixA}

Piezas diseñadas e impresas para el desarrollo. El prototipo del sistema previo a la versión final (funcionando en el SAT de Rivas) utiliza estas piezas.

\newpage

\includepdf[pages=-, pagecommand={}]{Appendices/externos/piezas.pdf}
