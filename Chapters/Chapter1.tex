% Chapter Template

\chapter{Introducción} % Main chapter title

\label{Chapter1} % Change X to a consecutive number; for referencing this chapter elsewhere, use \ref{ChapterX}

%----------------------------------------------------------------------------------------
%	SECTION 1
%----------------------------------------------------------------------------------------

\section{Objetivo}

Diseño y construcción de un sistema que forma parte de la cadena de montaje de un juguete comercial: el robot educativo Zowi. La principal función de este sistema es la de ajustar, de forma automática, las posiciones de los servomotores del juguete. Adicionalmente, se implementan otras funcionalidades importantes tal como la descarga del software final en el controlador del robot.

\subsection{Requisitos}

Algunos de los requisitos marcaron el camino a seguir hasta la solución final, facilitando la toma de decisiones en diferentes puntos del proyecto. Las peticiones más relevantes para el diseño fueron las siguientes:
\begin{itemize}
  \item Plazo de finalización fijado en 2 meses.
  \item Replicable fácilmente; a poder ser, por terceros.
  \item Utilizable por personal con poca o ninguna formación técnica.
  \item Cadencia aproximada de la línea de producción: 30-60 segundos.
  \item Deseable: fácil instalación.
\end{itemize}

\section{Estructura del proyecto}

La información se presenta de la siguiente forma:

\begin{itemize}
  \item En este Capítulo \ref{Chapter1} se presenta una breve introducción del proyecto.
  \item En el Capítulo \ref{Chapter2} se describe el motivo del proyecto, una pequeña descripción de las posibles soluciones y una base teórica sobre los principios y componentes más importantes del sistema, así como una pequeña mención a las tecnologías y herramientas que han sido útiles o necesarias para su desarrollo.
  \item El Capítulo \ref{Chapter3} se centra en la línea de desarrollo, mostrando las diferentes etapas y prototipos por los que se ha pasado hasta llegar a la versión final, con una descripción de las funciones de los componentes electrónicos dentro del sistema y del software creado o utilizado.
  \item En el Capítulo \ref{Chapter4} se sintetizan los resultados.
  \item En los anexos se recoge gran cantidad de la información del proyecto, conteniendo código, planos, esquemáticos o tablas de gran tamaño, entre otros. Será frecuente el uso de referencias a los anexos durante todo el documento.
\end{itemize}

Para el desarrollo de este documento, se ha generado un repositorio en Github con dirección en \url{https://github.com/uborzz/proyecto-ind}. En este repositorio se puede encontrar material adicional relacionado con el proyecto que no ha sido anexado, tal como datasheets, anotaciones, datos para análisis o vídeos de los prototipos, entre otras cosas. 
